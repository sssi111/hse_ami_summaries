\documentclass[12pt]{article}

\usepackage{cmap}
\usepackage[T2A]{fontenc}
\usepackage[utf8]{inputenc}
\usepackage[english, russian]{babel}
\usepackage{amsthm,amsmath,amssymb,tabto}
\usepackage{hologo}
\newtheorem{definition}{Definition}
\newcommand{\floor}[1]{\left\lfloor #1 \right\rfloor}
\newcommand{\ceil}[1]{\left\lceil #1 \right\rceil}

\usepackage{listings}
\usepackage{color}
\usepackage{xcolor}

\title{Конспект по алгебре 2 семестр}


\begin{document}
\maketitle
\section{Векторные пространства}
\begin{definition}
K - поле (скаляры - чиселки), векторное пространстов над K - это тройка (V + $\cdot$), где плюсик и точечка это бинарные опреции. Сложение $V + V \Rightarrow V$, умножение (на скаляр) $K \cdot V \Rightarrow V$
\end{definition}

Аксиомы $(V +)$ - абелева группа: \begin{enumerate}
    \item $\overline{0}$ - нейтральный элемент 
    \item $-v$ обратный по сложению
    \item Коммутативность по сложению
    \item Ассоциативность по сложению
    \item $(K_1 * K_2) * v = K_1 * ( K_2 * v)$ - согласованность операций (тут звёздочки немного разное умножение - сначала перемножение скаляров, потом домножение на скаляр)
    \item $K ( V_1 + V_2) = K V_1 + K V_2$
    \item $(K_1 + K_2) V = K_1V + K_2V$
    \item $1 * V_1 = V_1$
\end{enumerate} 
Для аксиом $\forall K1,\  K2 \in K, \ , \forall V_1, \ V_2 \in V$
\\

Замечания:
\begin{enumerate}
    \item  $0 * v = 0$
    \item $(-1) v = -v$ 
    \item $a + b = b + a$ - коммутативность следует из верхних 8-ми
\end{enumerate}
Доказательства замечаний: 
\begin{enumerate}
    \item \textbf{TODO}
    \item $(-1) v + v = (-1 + 1)v = 0 v = 0$, значит $(-1)v$ - обратный к $v$
    \item \textbf{TODO}
\end{enumerate}
\\

\textbf{Пример:}  
\begin{enumerate}
    \item Вектора на плоскости (классы эквивалентности направленных отрезков) - можем сопоставить каждому вектору координаты начала конца и на координатах определить операции
    \item Арифметическое векторное пространство - $$K^n = \{ \begin{pmatrix}
    K_1 \\
    K_2\\
    ... \\
    K_n
  \end{pmatrix} |\  \forall K_i \in K\}$$ на такой красоте можем определить умножение, и домножение на скаляр, для них работают аксиомы (не верите - проверите)
  \item $ ^nK = \{ K_1, K_2, ..., K_n \}$ 
  
  $K^n$ и $ ^nK$ - изоморфны, $\exists$ биективный гомоморфизм
\end{enumerate} 

 



\begin{definition}
    u, v - векторные простарнства из K, $u \rightarrow v$ - гоморфизм (линейное отображение) - значит:
    \begin{enumerate}
        \item f(a + b) = f(a) + f(b)
        \item f(ka) = kf(a) 
        \item отображения это геометрическая конструкция
    \end{enumerate}
\end{definition} 


\textbf{Больше примеров Богу примеров:}
\begin{enumerate}
    \item $К[x]$ - векторное пространство над K
    \item $K[x^n]=\{f\ |\ deg\ f\leq n\}$ - векторное пространство над K
    \item $K[[x]]$ - векторное пространство над K
    \item $R$ - коммутативное кольцо, $K \subset R$ - поле(подкольцо), тогда $R$ - в.п. над $K$
    \item $\mathbb{C}$ - в.п над $\mathbb{R}$
    \item $\mathbb{R}$ - в.п над $\mathbb{Q}$
\end{enumerate}

\textbf{Ещёёёё примеры} \begin{enumerate}
    \item M - множество, V := {f : M \rightarrow K} - векторное пространство над $K$, потому что $(f_1+f_2)(m):=f_1(m)+f_2(m)$, $(k\cdot f)(m):=k\cdot f(m)$ - я не понимаю
    \item $func(\mathbb{R}, \mathbb{R})$ - векторное пространство, $func_c(\mathbb{R}, \mathbb{R})$ - тоже векторное над $\mathbb{R}$, но непрерывное
    \item $(a_1, a_2, ..., a_n) \in R$ будем называть Фибоначчиевой, если $\forall i \ a_i = a_{i - 1} + a_{i - 2}$. Для них справедливо что $a_i + b_i$ и $ka_i$ - тоже фибоначчиевы, значит в.п.
    \item множество вычетов по модулю 2 ($\mathbb{Z}_{/2\mathbb{Z}}$, сумму определим как симметрическую разность ($X+Y:=(X\cup Y)\setminus (X\cap Y)$), умножение как $0x = 0,\ 1x = x$, получили векторное пространство. \textbf{Замечание} важно что модуль 2, потому что (1 + 1) x = x + x = 0 в K, тогда char(k) = 2
\end{enumerate}
\\

\textbf{Как ввести координаты в наши примеры:}
\begin{enumerate}
    \item $K[x]_n = \{a_0, a_1, .. , a_n\} = (a_1, ..., a_n)$ $K[x]_n \cong K^{n + 1}$
    \item $\mathbb{C}$ над $\mathbb{R}$ $z = a + bi$, $\mathbb{C} \cong \mathbb{R}^2$
    \item Пример плохих координат: $z \rightarrow (r, \phi)$ - не согласуется с операциями
    \item $Func(\mathbb{N}, \mathbb{R})$ - $\mathbb{R}^{n}$ тут n типа $\infty$
    \item любая фибоанчиева кодируется первыми двумя номерами, а значит $(a_1, a_2) \cong \mathbb{R}^2$ 
    \item Подмножества это бинарная строка $N \subset M$, $N = (\epsilon_1, ..., \epsilon_n), \ \forall \epsilon_i = $
 \begin{cases}
   1 & $a_i \in M$\\
   0 & $a_i \notin M$
 \end{cases}
\end{enumerate} 



\section{Базис и размерность} 
\begin{definition}
    $v_1, .. v_n \in V$ - векторное пространство над $K$, a-шки это скаляры ($\forall a_i \in K$), тогда $a_1v_1 + a_2v_2 + ... + a_nv_n$ - ленейная комбинация $v_1, ..., v_n$ с коэффициентами $a_i$. 
\end{definition}


\textbf{Замечание} множество лениейных комбинаций ($v_1, v_2, ..., v_n \in V$) замкнутое отностительно (+ $\cdot$)  является линейным пространством. Это линейная оболочка $v_1, v_2, ..., v_n = \langle v_1, ..., v_n \rangle$, $\sum a_iv_i+\sum b_iv_i=\sum (a_i+b_i)v_i;\ k\cdot \sum a_iv_i=\sum (ka_i)\cdot v_i$ \\


\textbf{Замечание 2} всё тоже можно представить для бесконечных систем, но тогда опасно определять ленейную комбинацию (для этого говорим что конечное число не нулей)
\\

\textbf{Примеры} Для одного вектора линейная оболочка это прямая, для двух это плоскость (если не колинеарны) 


\begin{definition}
    Условия ленейно независимости:
    \begin{enumerate}
        \item  Никакой вектор не выражается через остальные $\forall i\ v_i\neq \sum_{j\neq i} {a_jv_j}$
        (прикол - пустое множество это 0, один нулейовй вектор линейно зависим)
        \item Никакая нетривиальная линейная комбинация не равна нулю $\sum_{j} {a_jv_j}=0\Rightarrow \forall i\ a_i=0$
    \end{enumerate}

\end{definition}
 

\textbf{Докажем равносильность} - очень очев: 

$2 \Rightarrow 1$ пусть $v1 = \sum_{\forall i} {a_iv_i}$ и не все нуль, тогда перенесём и на обратные поменяем. 

$1 \Rightarrow 2$ надо перенести и поделить на коэфициент v_1=-\frac{a_2}{a_1}-... 

(тут важно что k это поле, потому что делим на $a_1$ из поля и можем так делать) 

\textbf{Замечание} Можно сделать такое для ассоциативного кольца (с 8 аксиомами) и получим $(V + *)$ - называется R-модулем

\begin{definition}
    V - векторное пространство над K, $\langle\{v_i\} \rangle =V$ - порождаяющая V система (как идеал)
\end{definition} 
\\

\textbf{Замечание} - пусть $M \subset V$ - подмножество векторов, если M было линейно независимое, то $N \subset M$ - линейно независимое, и если M - порождающая, то $M \subset N$ N - порождающая
\\

\begin{definition}
    $\{v_i\} \subset V$, $V$ - векторное пространство, $\{v_i\}$ - базис, если выполнены 4 равносильных условия
    \begin{enumerate}
        \item Линейно независимая и порождающая
        \item Максимальная линейно независимая
        \item $\{v_i\}$ - минимальное порождающее
        \item $\forall v \in V$ - представляется единственным образом как линейная комбинация $\{v_i\}$ 
    \end{enumerate}

\end{definition} 

\textbf{Доказательства:} План: $3 \Leftrightarrow 1 \Leftrightarrow 2$ и где-то добавить 4..

$1 \Rightarrow 2$ добавим любой вектор (комбинацию) получим зависимость

$2 \rightarrow 1$ $<\{v_i\}> = V$, написать по фотке

2) a = 0, тогда есть комбинация где не все скаляры 0, равная 0, но это противоречит линейной независимости 

$1 \rightarrow 4$ запишем двумя способами, вычтем и получим  что если линейно независимо (то все коэффициенты стали 0)

$4 \rightarrow 1$ лнз v1 = $\sum_{\forall i \noteq 1} {a_ivi}$, v_1 = 1 v_1 - противоречие единственности 


Определение - V - пространство, {v_i} - базис, для $\forall v \in V \exist {a_i}$ - координаты в базисе;;; $v \rightarrow ^nK$ - изоморфизм
д-во - биективность корректность гомоморфность 

Базис - строка, координаты - столбец

Пример с трёхчленом для базиса (1 , x, x^2), (x^2 + 1, x^2 + x + 3, x^2 - x) 


Существование базиса - v - называется конечно мерным - если в V существует конечная порождающая система. V - является линейной оболочкой конечного числа векторов.
Лемма - из любой конечно мерной системы можно извлечь базис
Теорема - из любого конечно мерного пространства можно извлечь базис

Доказательство леммы - доказать, что если один вектор выражается через остальные, то его можно убрать

Замечание (!!!) в любом пространстве есть базис, 

Лемма цорна(??)

Когда базис бесконечный - говорим что его нет

определение - V - векторное пространство(конечно мерное), размерность V (div(V)) - количество векторов в его базисе 


Факт(теорема) - в любых двух базисах поровну элементов. Доказательство - следует из леммы.

Лемма О линейной зависимости линейных комбинаций  (лзлк)

(u_1, .., u_n) - все u_i - линейные комбинации из (v_1, v_2, .. , v_m) и n > m, тогда ушки линейно зависимы - почему

доказательство - нуо n = m + 1, потому что умеем убирать лишнее, индукция по n - база - n = 2 u_1 = a_1v_1, u_2 = a_2v_1, если есть ноль - то очев зависимо, если нет, то второй выражается через первый

переход $n \rightarrow n + 1$ каждая u выражается через v.., пробуем исключить последнюю v_{n + 1}, если смогли, тогда по предположению победа, если не смогли (u_1 = ... + a_{n + 1}v_{n + 1}) - есть такой - у которого не нуль тогда повычетаем с константами чтобы последнее умерло 
\end{document}