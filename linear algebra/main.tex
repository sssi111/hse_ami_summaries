\documentclass[12pt]{article}

\usepackage{cmap}
\usepackage[T2A]{fontenc}
\usepackage[utf8]{inputenc}
\usepackage[english, russian]{babel}
\usepackage{amsthm,amsmath,amssymb,tabto}
\usepackage{hologo}
\newtheorem{definition}{Definition}
\newcommand{\floor}[1]{\left\lfloor #1 \right\rfloor}
\newcommand{\ceil}[1]{\left\lceil #1 \right\rceil}

\usepackage{listings}
\usepackage{color}
\usepackage{xcolor}

\title{Конспект по алгебре 2 семестр}


\begin{document}
\maketitle
\section{Векторные пространства}
\begin{definition}
K - поле (скаляры - чиселки), векторное пространство над K - это тройка (V + $\cdot$), где плюсик и точечка это бинарные операции. Сложение $V + V \Rightarrow V$, умножение (на скаляр) $K \cdot V \Rightarrow V$
\end{definition}

Аксиомы: $(V +)$ - абелева группа: \begin{enumerate}
    \item $\overline{0}$ - нейтральный элемент 
    \item $-v$ обратный по сложению
    \item Коммутативность по сложению
    \item Ассоциативность по сложению
    \item $(k_1 * k_2) * v = k_1 * (k_2 * v)$ - согласованность операций (тут звёздочки немного разное умножение - сначала перемножение скаляров, потом домножение на скаляр)
    \item $k (v_1 + v_2) = kv_1 + k v_2$
    \item $(k_1 + k_2) v = k_1v + k_2v$
    \item $1_K * v_1 = v_1$ (тут и в дальнейшем под $0_V$, $1_K$ и в прочих подобных записях имеется в виду 0 или 1 из соответствующего множества)
\end{enumerate} 
Для аксиом $\forall k1,\  k2 \in K, \ , \forall v_1, \ v_2 \in V$
\\

Замечания:
\begin{enumerate}
    \item  $0_K * v = 0_V$
    \item $(-1) v = -v$ 
\end{enumerate}
Доказательства замечаний: 
\begin{enumerate}
    \item $0_K * v = (0_K + 0_K) * v = 0_K * v + 0_K * v$, и т.к. $V$ - абелева, то вычтем обратное к $0_K * v$ из обеих частей и получим $0_K * v = 0_V$
    \item $(-1) v + v = (-1 + 1)v = 0_K*v = 0_V$, значит $(-1)v$ - обратный к $v$
\end{enumerate}
\\

\textbf{Пример:}  
\begin{enumerate}
    \item Вектора на плоскости (классы эквивалентности направленных отрезков) - можем каждый вектор отложить от начала координат и на отложенных векторах определить операции
    \item Арифметическое векторное пространство - $K^n = \{ \begin{pmatrix}
    K_1 \\
    K_2\\
    ... \\
    K_n
  \end{pmatrix} |\  \forall K_i \in K\}$ на такой красоте можем определить умножение, и домножение на скаляр, для них работают аксиомы (не верите - проверите)
  Кста, еще это в.п. называется $n$-мерным арифметическим пространством столбцов, (но использоваться это название будет редко либо не будет)
  \item $ ^nK = \{ K_1, K_2, ..., K_n \}$ (пространство строк)
  
  $K^n$ и $ ^nK$ - изоморфны, $\exists$ биективный гомоморфизм
\end{enumerate} 

 



\begin{definition}
    U, V - векторные пространства над K, $f$ : $U \rightarrow V$ - гомоморфизм (линейное отображение) - значит:
    \begin{enumerate}
        \item f(a + b) = f(a) + f(b)
        \item f(ka) = kf(a) 
        \item отображения это геометрическая конструкция
    \end{enumerate}
    ($ \forall$ $ a, b \in U$, $f(a), f(b) \in V$, $k \in K$)
\end{definition} 


\textbf{Больше примеров Богу примеров:}
\begin{enumerate}
    \item $K[x]$ - векторное пространство над K
    \item $K[x^n]=\{f\ |\ deg\ f\leq n\}$ - векторное пространство над K
    \item $K[[x]]$ - векторное пространство над K
    \item $R$ - коммутативное кольцо, $K \subset R$ - поле(подкольцо), тогда $R$ - в.п. над $K$
    \item $\mathbb{C}$ - в.п над $\mathbb{R}$
    \item $\mathbb{R}$ - в.п над $\mathbb{Q}$
\end{enumerate}

\textbf{Ещёёёё примеры} \begin{enumerate}
    \item M - множество, $V := {f : M \rightarrow K}$ - векторное пространство над $K$, потому что $(f_1+f_2)(m):=f_1(m)+f_2(m)$, $(k\cdot f)(m):=k\cdot f(m)$ (отобразили элемент из $M$ в $K$, получили элемент $K$, домножили на другой элемент из $K$)
    \item $func(\mathbb{R}, \mathbb{R})$ - векторное пространство, $func_c(\mathbb{R}, \mathbb{R})$ - тоже векторное над $\mathbb{R}$, но непрерывное
    \item $(a_1, a_2, ..., a_n) \in R$ будем называть Фибоначчиевой, если $\forall i > 2 \ a_i = a_{i - 1} + a_{i - 2}$. Для них справедливо что $a_i + b_i$ и $ka_i$ - тоже фибоначчиевы, значит в.п.
    \item $V$ - множество подмножеств, $K$ - множество вычетов по модулю 2 ($\mathbb{Z}{/2\mathbb{Z}}$), $+_V$ определим как симметрическую разность ($X+Y:=(X\cup Y)\setminus (X\cap Y)$), умножение как $0*X = \emptyset,\ 1*X = X$, получили векторное пространство. \textbf{Замечание} важно что модуль 2, потому что $(1 + 1) X = X + X = 0$ в $K$, тогда $char(k) = 2$
\end{enumerate}
\\

\textbf{Как ввести координаты в наши примеры:}
\begin{enumerate}
    \item $K[x]_n = \{a_0, a_1, .. , a_n\} = (a_0, a_1, ..., a_n)$ $K[x]_n \cong K^{n + 1}$
    \item $\mathbb{C}$ над $\mathbb{R}$ $z = a + bi$, $\mathbb{C} \cong \mathbb{R}^2$
    \item Пример плохих координат: $z \rightarrow (r, \phi)$ - не согласуется с операциями
    \item $Func(\mathbb{N}, \mathbb{R})$ - $\mathbb{R}^{n}$ тут n типа $\infty$
    \item любая фибоанчиева кодируется первыми двумя номерами, а значит $(a_1, a_2) \cong \mathbb{R}^2$ 
    \item Подмножества это бинарная строка $N \subset M$, $N = (\epsilon_1, ..., \epsilon_n), \ \forall \epsilon_i = 
 \begin{cases}
   1 & a_i \in M\\
   0 & a_i \notin M
 \end{cases}$
\end{enumerate} 



\section{Базис и размерность} 
\begin{definition}
    $v_1, .. v_n \in V$ - векторное пространство над $K$, a-шки это скаляры ($\forall a_i \in K$), тогда $a_1v_1 + a_2v_2 + ... + a_nv_n$ - линейная комбинация $v_1, ..., v_n$ с коэффициентами $a_i$. 
\end{definition}


\textbf{Замечание} множество линейных комбинаций ($v_1, v_2, ..., v_n \in V$) замкнуто относительно (+ $\cdot$), т.е. является векторным пространством. Это линейная оболочка $v_1, v_2, ..., v_n = \langle v_1, ..., v_n \rangle$, $\sum a_iv_i+\sum b_iv_i=\sum (a_i+b_i)v_i;\ k\cdot \sum a_iv_i=\sum (ka_i)\cdot v_i$ \\


\textbf{Замечание 2} всё тоже можно представить для бесконечных систем, но тогда опасно определять линейную комбинацию (для этого говорим что конечное число не нулей). Мы в основном будем говорить про конечные системы
\\

\textbf{Примеры} Для одного вектора линейная оболочка это прямая, для двух это плоскость (если они не коллинеарны) 


\begin{definition}
    Условия линейной независимости системы векторов $v_1, v_2, ..., v_n \in V$:
    \begin{enumerate}
        \item  Никакой вектор не выражается через остальные $\forall i\ v_i\neq \sum_{j\neq i} {a_jv_j}$ ($\forall a_i$)
        (прикол - сумма по пустому множеству это $O_V$ , один нулевой вектор линейно зависим)
        \item Никакая нетривиальная линейная комбинация не равна нулю $\sum_{j} {a_jv_j}=0\Rightarrow \forall i\ a_i=0$
    \end{enumerate}

\end{definition}
 

\textbf{Докажем равносильность} - очень очев: 

$2 \Rightarrow 1$ пусть $v_i = \sum_{\forall j \neq i} {a_jv_j}$, тогда перенесем $v_i$ в сумму и получим нетривиальную линейную комбинацию с нулевой суммой (нетривиальна потому что при $v_i$ коэффициент $-1$), противоречие

$1 \Rightarrow 2$ пусть существует нетривиальная комбинация, тогда перенесем какой-нибудь $v_i$ с ненулевым коэффициентом в другую часть равенства и поделим обе части равенства на коэффициент при $v_i$, выразили $v_i$ через остальные, противоречие.

(тут важно что k это поле, потому что делим на $a_i$ из поля и можем так делать) 
В дальнейшем может использоваться сокращение л.н.с. - линейно независимая система и аналогично л.з. - линейно зависимая.

\textbf{Замечание} Можно сделать такое для ассоциативного кольца (с 8 аксиомами) и получим $(V + *)$ - называется R-модулем (или модуль над $R$)

\begin{definition}
    V - векторное пространство над K, $v_1, v_2, ..., v_n \in V$ - система векторов, тогда она называется порождающей если $\langle\{v_i\} \rangle =V$
\end{definition} 
\\

\textbf{Замечание} - пусть $M \subset V$ - система векторов, если M - л.н.с, то $N \subset M$ - л.н.с, и если M - порождающая, и $M \subset N$, то $N$ - порождающая
\\

\begin{definition}
    $\{v_i\} \subset V$, $V$ - векторное пространство, $\{v_i\}$ - базис, если выполнены 4 равносильных условия
    \begin{enumerate}
        \item $\{v_i\}$ - л.н.с. и порождающая
        \item $\{v_i\}$ - максимальная л.н.с. (добавим хоть 1 любой вектор из $V$ - и станет зависимой)
        \item $\{v_i\}$ - минимальная порождающая (уберем хоть 1 вектор - перестанем быть порождающей)
        \item $\forall v \in V$ - представляется единственным образом как линейная комбинация $\{v_i\}$ 
    \end{enumerate}
    p.s. в общем-то примерно так же работает и на бесконечных системах, но стоит помнить про тонкости с линейной комбинацией, и опять же, мы с таким работать почти не будем.

\end{definition} 

\textbf{Доказательства:} План: $3 \Leftrightarrow 1 \Leftrightarrow 2$ 

$1 \Rightarrow 2$ добавим любой вектор, он выражается через предыдущие т.к. система порождающая, следовательно перестали быть л.н.с., ч.т.д. (т.е. текущая система - максимальная л.н.с.)

$2 \Rightarrow 1$ пусть мы не порождающая система, тогда мы не можем выразить из текущих векторов какой-то вектор $u \in V$, добавим его в систему, получим все еще л.н.с., противоречие с максимальностью

$1 \Rightarrow 4$ Порождающая $\Rightarrow$ каждый вектор можно выразить хотя бы одним способом. Пусть какой-то можно выразить 2-мя разными способами, тогда его разность с самим собой (т.е. $O_V$) можно представить как линейную комбинацию с нетривиальным набором коэффициентов, противоречие с л.н.с.

$4 \Rightarrow 1$ любой вектор представляется хотя бы 1 способом $\Rightarrow$ система порождающая. Л.н.с.: пусть мы линейно зависимы, тогда воспользуемся первой трактовкой л.н.с.: какой-то вектор из нашей системы выражается через остальные. Тогда противоречие с единственностью: с одной стороны, этот вектор выражается просто как он сам, с другой стороны, как линейная комбинация остальных векторов из системы.

$3 \Rightarrow 1$ Порождаемость уже есть, пусть мы л.з. Тогда уберем какой-нибудь вектор, выражающийся через остальные (остались порождающей системой т.к. там где раньше участвовал удаленный вектор подставим его выражение через остальные вектора). Противоречие с минимальностью порождаемости

$1 \Rightarrow 3$ порождаемость есть, пусть не минимальная, тогда можно удалить какой-нибудь элемент, не потеряв порождаемость (т.е. этот элемент выражается через остальные). Противоречие с л.н.с.


\begin{definition}
    V - пространство, $\{v_i\}$ - базис, $\forall v \in V \exist {a_i}$ - координаты в базисе, тогда $V \Rightarrow ^nK$ - изоморфизм
\end{definition} 
\textbf{Доказательство:} биективность, коректоность, гомоморфоность - \textbf{TODO}
\\

\textbf{Замечание по оформлению} Базис - строка, координаты - столбец

\textbf{Пример, что разложения зависят от базиса} Базисы: $(1, x, x^2),\ (x^2 + 1, x^2 + x + 3, x^2 - x)$, тогда $x^2 + 1$ это $(1, 0, 1)$, или $(1, 0, 0)$. \textbf{TODO - написать столбики}. \\


\textbf{Существование базиса}
\begin{definition}
    $V$ - называется конечно мерным - если в $V$ существует конечная порождающая система $(V=\langle v_1,...,v_n \rangle)$. $V$ - является линейной оболочкой конечного числа векторов.
\end{definition}

\textbf{Лемма} из любой конечно мерной системы можно извлечь базис.

\textbf{Теорема} из любого конечно мерного пространства можно извлечь базис

\textbf{Доказательство леммы} Рассмотрим конечно мерную порождающую систему $\langle a_1, ..., a_n \rangle$, пусть она л.з., тогда какой-то вектор (для определённости $a_n$ - выражается через $a_1, ..., a_{n - 1}$, тогда $\langle a_1, ..., a_{n - 1} \rangle$ - порождающая. Продолжим, так как система конечно мерная, то процесс завершится - получим базис. \\

\textbf{Замечание: }Когда базис бесконечный - говорим что его нет.

\begin{definition}
    V - векторное пространство(конечно мерное), размерность $V$ $(div(V))$ - количество векторов в его базисе 
\end{definition} 


\textbf{Факт(теорема)} - в любых двух базисах поровну элементов. 
\\

\textbf{Доказательство факта} - следует из леммы.
\\

\textbf{Лемма О линейной зависимости линейных комбинаций  (лзлк)}

Пусть $(u_1, .., u_n)$ - все $u_i$ - линейные комбинации из $\langle v_1, v_2, .. , v_m\rangle$ и $n > m$, тогда $u$ линейно независимы.

\textbf{Доказательство} - нуо n = m + 1, потому что можем убрать какие-то, индукция по n - база - n = 2 u_1 = a_1v_1, u_2 = a_2v_1, если есть ноль - то очев зависимо, если нет, то второй выражается через первый

переход $n \rightarrow n + 1$ каждая u выражается через v.., пробуем исключить последнюю v_{n + 1}, если смогли, тогда по предположению победа, если не смогли (u_1 = ... + a_{n + 1}v_{n + 1}) - есть такой - у которого не нуль тогда повычетаем с константами чтобы последнее умерло 
\\

\textbf{Лемма}  $u$ - конечномерное векторное пространсвто $u_1, u_2, .. u_k \in u$ лнз система, тогда $\exists u_{k + 1}, ..., u_n \notin u$ такое что их объединение это базис u.

\textbf{Доказательство:} пусть есть k векторов - если максимально линейно независимы(базис), либо добавим ещё один вектор. Будем добавлять, если их стало n+1, то по лнзлк она линейно зависима. Значит когда добавим ровно до n будет базис.
\\

\textbf{Следствие леммы:} если $U, V$ - веткорные пространства над $K$(конечномерные), $U \leq V \Rightarrow dim(U) \leq dim(V)$.

\textbf{Доказательство:} $u_1, ..., u_k$ - базис u, дополним до базиса v, знак очевиден так как подпространство.

\textbf{Применение:} $\mathbb{Q}, \alpha$ - алгебраическое, если $\exists f \in \mathbb{Q}[x]: f(\alpha) = 0$

\textbf{Пример:} $\sqrt[7]{3}$ - корень $x^7 - 3 = 0$. 
\\

\textbf{Теорема} $\alpha$ - алгебраическое, $ p \in Z[x]$, то  $p(\alpha)$ - алгебраическое.

\textbf{Доказательство:} Рассмотрим векторное пространство над $\mathbb{Q}$, получим множество $\{p(\alpha) | p \in Q[x]\}$ - замкнуто относительно сложения и умножения на $\mathbb{Q}$.

Получили векторное пространство (оно конечно мерное), потому что $\exists a_{n - 1}, ..., a_0, \in \mathbb{Q} \ \alpha^n = \sum {a_i \cdot \alpha^{i + 1}}$ Домножим на $\alpha$ - можем выразить $\alpha^j$ через первые n степеней (начиная с 0), получается что размерность пространства ($dim(V_{\alpha}) \leq n)$. 

$\exists p \in \mathbb{Z}[x]$ $1, p(\alpha), ..., p(\alpha)^n \ in V_{\alpha}$ - они линейно зависимы так как их n+1, тогда $\exists q_0, ..., q_n$ - $p(\alpha)$ - это корень многочлена $q_nx^n + ... + q_1x + q_0$
\\

\textbf{Теорема} Пусть V коненчно мерное векторное пространство над K, тогда $\exists !n : V \cong K^n$. 

\textbf{Доказательство:}

единственность: $n = dim(K^n) = dim(V)$ 

существованеи - пусть $u_1, ..., u_n$ - базис V, рассмотрим $i: K^n \rightarrow V$ $\begin{pmatrix}
    a_1 \\
    ... \\
    a_n
\end{pmatrix} \rightarrow (a_1v_1 + ... + a_nv_n)$. Это биекция так как $u_1, ..., u_n$ - базис ($\forall u, \exists (a_1, ..., a_n) \ \ u = \sum a_i u_i$.
\\

\textbf{Обозначение} $\mathrm{U} = \begin{pmatrix}
    a_1 \\
    a_2 \\
    ... \\
    a_n 
\end{pmatrix} = [u]_{\{u_i\}}$ - столбец координат в базисе $\{ u_i \}$
\\

\textbf{Пример:} фибоначчиева последовательность - есть базис $(1, 0, 1, 1, 2 ...)$, $(1, 1, 2, 3, ... )$ $(a, b, a + b, ...)$ = $av_1 + b v_2$. Прикольный базис: $(1, \phi, \phi^2, ...)$, $(1, -\frac{1}{\phi}, (-\frac{1}{\phi})^2)$ - если решить в этом базисе, получим формулу Бине).
\\

\section{Системы линейных уравнений}
$S:\begin{cases}
    a_{11}x_1 + ... + a_{1m}x_m = b_1 \\
    ... \\
    a_{n1}x_1 + ... + a_{nm}x_m = b_n
 \end{cases}$ 
 \\
 $$A_i = \begin{pmatrix}
     a_{1i} \\
     a_{2i} \\
     ... \\
     a_{ni}
 \end{pmatrix} \in K^n, \ \ B = \begin{pmatrix}
     b_1 \\
     b_2 \\
     ... \\
     b_n
 \end{pmatrix}$$ Если S имеет решение, то $B \in \langle A_1, ..., A_n \rangle$
 \\
 \begin{definition}
     Однородная слу - B = 0. У неё всегда есть тривиальное решение $x_1 = x_2 = ... = x_n = 0$
 \end{definition}

 \textbf{Замечание} Если у ослу есть нетривиальное решение, то $A_1, ..., A_n$ - линейно зависимы.

 \textbf{Частный случай:} $m > n$, тогда $A_1, ..., A_{n + 1} \in K^n \Rightarrow$ лз и есть нетривиальное решение.

 \begin{definition}
     A - абелева группа, $I, J$ - конечные множества, тогда матрица над A это отображение $I \times J \rightarrow A \ | \ (i, j) \rightarrow a_{ij} \in A$
 \end{definition}
 Часто $I = \{1, ..., n\}, \ J = \{1, ..., m\}$, в таком случае множество всех матриц записывается как $M_{n,m}(A)$
\\

Определим сложение (если размеры двух матриц равны $a_{ij} + b_{ij} = (a + b)_{ij}$ - можно проверить всякие коммутативности, ассоциативности. Тогда $M_{n,m}$ - абелева группа.
\\

Пусть $A = R$ - кольцо, тогда определим умножение матриц:
\begin{enumerate}
    \item $M_{1,m} \times M_{m, 1} \rightarrow R$, $$(a_1, ..., a_m) \cdot \begin{pmatrix}
        b_1 \\
        ... \\
        b_n
    \end{pmatrix} = a_1b_1 + ... a_nb_n$$
    \item $M_{k,m} \times M_{m, 1} \rightarrow M_{k, 1}$, $$\begin{pmatrix}
        row_1 \\
        row_2 \\
        ... \\
        row_k
    \end{pmatrix} \cdot \begin{pmatrix}
        b_1 \\
        b_2 \\
        ... \\
        b_m
    \end{pmatrix} = \begin{pmatrix}
        row_1 \cdot column \\
        row_2 \cdot column \\
        ... \\
        row_3 \cdot column
    \end{pmatrix} \in M_{k, 1}(R) = R^k$$
    \item $M_{k,m} \times M_{m, l} \rightarrow M_{k, l}$, $A(c_1 | c_2 | ... | c_l)$, $$c_{ij} = \sum_{s = 1}^m {a_{is}b_{sj}}$$
\end{enumerate}

СЛУ $A*X = B, \ X = A^{-1}B$ - но обратная матрица есть редко.

\textbf{Свойства матриц:}
\begin{enumerate}
    \item $A_{mk} \cdot (B+C)_{kl} = AB + AC$, тоже самое для правоассоциативной $(B+C)_{kl} \cdot A_{mk}  = BA + CA$
    \item $(AB)C = A(BC)$ - если вообще можно перемножать (см размер).
\end{enumerate}
\textbf{Доказательство:} \begin{enumerate}
    \item \textbb{TODO}
    \item $$((AB)C)_{ij} = \sum_{s = 1}^m {(AB)_{is}C_{sj}} = \sum_{s = 1}^m {(\sum_{t = 1}^l {A_{it}B_{ts}})C_{sj}}$$
    $$(A(BC))_{ij} = \sum_{s = 1}^m {A_{is}(BC)_{sj}} = \sum_{s = 1}^m {A_{is}(\sum_{t = 1}^l {B_{st}C_{tj}})}$$ - вроде равны.
\end{enumerate}

\textbf{Утверждение} $M_{n,n}(R)$ - ассоциаттивное кольцо с 1. 

\textbf{Доказательство:} Умножение определено, ассоциативность, дистрибутивность (абелева группа по сложению), еденица $\begin{pmatrix}
    1 & 0 & 0 & 0 \\
    0 & 1 & 0 & 0 \\
    0 & 0 & 1 & 0 \\
    0 & 0 & 0 & 1
\end{pmatrix}.
$EA = A \ \forall A \in M_{n, m}(R)$.

Замечание - умножение не коммутативно, обычно просто не сходится по размеру, если сходится, то пример: $$\begin{pmatrix}
    0 & 1 \\
    0 & 0
\end{pmatrix} \cdot \begin{pmatrix}
    0 & 0 \\
    1 & 0
\end{pmatrix} = \begin{pmatrix}
    1 & 0 \\
    0 & 0
\end{pmatrix}$$
$$\begin{pmatrix}
    0 & 0 \\
    1 & 0
\end{pmatrix} \cdot \begin{pmatrix}
    0 & 1 \\
    0 & 0
\end{pmatrix} = \begin{pmatrix}
    0 & 0 \\
    0 & 1
\end{pmatrix}$$

\\

\textbf{Замечание} R = k-поле
$M_n(k)$ - вп над K. $k(a_{i, j}) = ka_{i, j}$ Самый просто базис - это по одной 1 в каждом "векторе"

Умножение в $M_n(K)$ достаточно задать на базисе. $E_{ij} \cdot E_{kl} = \begin{cases}
   E_{il} & j \neq k\\
   0 & j \neq k
 \end{cases}$ 

$V$ - n  мерное пространство $v_1, ..., v_n$ - старый базис, $v_1', ... , v_n'$ - новый базис. $a_1, .., a_n \in k$ $(v_1, ..., v_n) \cdot \begin{pmatrix}
    a_1 \\
    a_2\\
    ... \\
    a_n
  \end{pmatrix} = a_1v_1 + ... + a_nv_n$

  \textbf{Матрица перехода} от $v_i$ к $v_i'$

  Пусть $x \in V$ $X = [x]_{\{v_i\}}$ Тогда новые координаты это старый координатный столбец умножить на матрицу перехода. 

  Доказательство: $(u_1', ..., u_n')(CX) = ((u_1', ..., u_n')C)X = (v_1, ..., v_n)X$, чтд
  \\

 \textbf{Следствие:} C - матрица перехода $(u_i) \rightarrow (u_i')$, $C'$ - матрица перехода $(u_i') \rightarrow (u_i'')$, тогда $C'C$ - матрица перехода $(u_i) \rightarrow (u_i'')$

 \textbf{Доказательство:} $\forall$ столбца X $\leftrightarrow \ x \in V$ $CX$ - кординаты в $(u_i')$, $C'(CX)$ кординаты в $(u_i'')$, тогда $C'C$ - матрица перехода.

 Частный случай - если u = u", тогда $C'  = C^{-1}$



 \section{Линейные отображения}

 \textbf{Напоминание:}$ U, V$ - вп над K, $A : U \rightarrow V$ линейно(гомотетия), если $A(u + \alpha v) = A(u) + \alpha A(v) \ \ \forall u, v \in U, \ \forall \alpha \in K$
\\

 \textbf{Пример:} $A(x) = x \ \ A(x) = 0$ - линейны

 \textbf{Пример:} $A: K^n \rightarrow K, \ dim(K) = 1, \ \langle1\rangle = K$ $$A\begin{pmatrix}
     x_1 \\
     ... \\
     x_n
 \end{pmatrix} = a_1x_1 + ... + a_nx_n$$
 Более общо: $A \in M_{m, n}, A(x) = Ax$ - линейное отображение $A: K^n \rightarrow K^m$, все линейные отображение - это домножения на матрицу.
 \\

 \textbf{Теорема} $(u_1, ..., u_n)$ - базис U, $v_1, ..., v_n \in V$ тогда $\exists ! \alpha: U \rightarrow V: u_i \rightarrow v_i$

  \textbf{Доказательство:}
  
  единственность: пусть $\alpha(u_i) = v_i = B(u_i)$ $$\forall u\in U: u=\sum a_iu_i\Rightarrow A(u)=A(\sum a_iu_i)=\sum a_iA(u_i)=\sum a_iB(u_i)=B(\sum a_iu_i)=B(u)$$
  
  Существование: $\forall u = \sum a_i u_i, \ A(u) = \sum a_i v_i$, тогда:
  
  $A(\alpha\cdot (\sum a_iu_i)+\sum b_iu_i)=A(\sum(\alpha\cdot a_i+b_i)u_i)=\alpha\cdot \sum a_iv_i+\sum b_iv_i=\alpha \cdot A(\sum a_iu_i)+A(\sum b_iu_i)
$  \\

  \begin{definition}
      $\alpha : U \rightarrow V$ $Ker(A) = \{ u \in U | A(u) = 0\}$ - ядро. Образ - $Im(A) = \{v \in V | \exists u \in U A(u) = v \}$
  \end{definition} 

  \textbf{Лемма} $Ker(A)$ - подпространство в $U$, $Im(A)$ - подпространство в $V$.

  Доказательство: (проверим замкнутость)
  
  1) $u, v \in Ker(A) \ A(u) = 0, A(v) = 0$ тогда $A(u + kv) = 0$

  2) $u = A(x),\ v = A(y)$, тогда $v + u = A(x) + A(y) = A(x + y) \rightarrow x + y \in Im$, $ku = kA(x) = A(kx) \rightarrow kx \in Im$
\\

\textbf{Примеры:} \begin{enumerate}
    \item A(x) = x: $Ker(A) = \{0\},\ Im(A) = V$
    \item A(x) = 0 $Ker(A) = V,\ Im(A) = \{0\}$
    \item v - вектор на декартовой плоскости, A(v) - проекция на $Ox$, тогда $Ker(A) = \langle \begin{pmatrix}
        0 \\
        1
    \end{pmatrix} \rangle,\ Im(A) = \langle \begin{pmatrix}
        1 \\
        0
    \end{pmatrix} \rangle$
\end{enumerate}

  \textbf{Теорема:} $A: U -> V$ - линейное отображение. $dim(Ker(A))+ dim(Im(A)) = dim(U)$
\\

  \textbf{Доказательство:} $dim(ker(A)) = m$ - $u_1, ..., u_m$ - базис ядра, дополним до базиса $U$. Применим линейное отображение $A(u_1), ... A(u_m)$ = 0
  
  \textbf{Утверждение:} $A(u_{m+1}), ... A(u_{n})$ - базис образа.
  
  \textbf{Докажем утверждение:} $A(u_{m+1}), ... A(u_{n})$ - лнз.

  Пусть $\alpha_{m + 1} A(u_{m + 1}) + ... + \alpha_{n} A(u_{n}) = 0$, тогда $A(\sum {\alpha_{m + i}u_{m + i}}) = 0 \rightarrow \sum {\alpha_{m + i}u_{m + i}} \in Ker(A)$, $\sum {\alpha_{m + i}u_{m + i}} = \sum {\alpha_{i}u_{i}}$ - вспомнили про базис $Ker$, но тогда $\sum {\alpha_{m + i}u_{m + i}} - \sum {\alpha_{i}u_{i}} = 0$, $\{u_i\}$ - лнз (базис U), тогда все $\alpha_i = 0$, получили базис образа, чтд.
\end{document} 
